Jon Brodziak, Matthew Supernaw, and Clay Porch~\newline
 N\+O\+A\+A, National Marine Fisheries Service~\newline
\hypertarget{index_Introduction}{}\section{Introduction}\label{index_Introduction}
~\newline
 The scale and connectivity of spatial processes are key to understanding ~\newline
 and predicting patterns in resource abundance in population and community ~\newline
 ecology. The importance of spatial scale and connectivity is a paradigm ~\newline
 for ecology (Levin 1992) and provides a strong rationale to focus on ~\newline
 metapopulations, or populations of populations, as the basic unit of population ~\newline
 ecology. To understand and predict impacts on populations, population ~\newline
 assessments should increasingly focus on the influence of spatial variation ~\newline
 on resources and the development of metapopulation assessments will require a ~\newline
 shift towards more complex and hierarchical population models.~\newline
 ~\newline
 From a software design perspective, an object-\/oriented design paradigm where one ~\newline
 can provide precise design specifications, have a shorter development phase for ~\newline
 new code, and expect easier maintenance with consistency and reusability through ~\newline
 time is a natural approach to developing more complex and hierarchical population ~\newline
 models. To some extent, such features are partially implemented in some existing ~\newline
 integrated assessment models, e.\+g. Stock Synthesis, but there is ample room for ~\newline
 improvements in model development, selection, testing, uncertainty quantification, ~\newline
 and assessment modeling components. In this context, rapid prototyping and testing ~\newline
 of alternative models for a diversity of fishery systems is desirable and could be ~\newline
 achieved through the ongoing development of a system library of tested modules and ~\newline
 templates. The capacity to build new models from object-\/oriented templates will ~\newline
 foster the ongoing development of structured assessment models and will also help ~\newline
 to avoid some of the life cycle issues of maintaining a single omnibus assessment model. \hypertarget{index_Purpose}{}\section{Purpose}\label{index_Purpose}
\hypertarget{index_System}{}\section{Overview}\label{index_System}
\hypertarget{index_System}{}\section{Overview}\label{index_System}
\hypertarget{index_Examples}{}\section{Examples}\label{index_Examples}
\hypertarget{MAS.hpp_References}{}\section{References}\label{MAS.hpp_References}
